\section{Prolegômenos}

\begin{frame}{EDOs como EDPs lineares}
	
	A partir do sistema de EDO:
	\begin{equation*}
		\frac{d}{dt} \phi(x,t) = R(\phi(x,t)), \quad \phi(x,0) = x
	\end{equation*}
	
	Define-se o Operador de Liouville:
	\begin{equation*}
		L = \sum_i R_i(x) \frac{\partial}{\partial x_i}
	\end{equation*}
	
	Assim, temos o sistema equivalente de EDP:
	\begin{equation*}
		u_t = Lu, \quad u(x,0) = g(x), \quad u(x,t) = g(\phi(x,t))
	\end{equation*}
	
\end{frame}
%--------------------------------------------------------

\begin{frame}{Notação de semigrupo}
	\begin{itemize}
		\item Usada para representar soluções de EDPs de forma compacta.
		\item Exemplo: equação do calor
		      \begin{equation*}
		      	v_t - \frac{1}{2}\Delta v = 0, \quad v(x,0) = \phi(x)
		      \end{equation*}
		\item Solução escrita como:
		      \begin{equation*}
		      	v(t) = e^{\frac{1}{2}t\Delta} \phi
		      \end{equation*}
		\item Propriedade: $e^{(t+s)\Delta} = e^{t\Delta} e^{s\Delta}$
	\end{itemize}
\end{frame}

%--------------------------------------------------------

\begin{frame}{Aplicando ao Operador de Liouville}
	\begin{itemize}
		\item Solução da EDP:
		      \begin{equation*}
		      	e^{tL}g(x) = g(\phi(x,t))
		      \end{equation*}
		\item $g$ comuta com a evolução do sistema.
		\item Relação de comutação:
		      \begin{equation*}
		      	Le^{tL} = e^{tL}L
		      \end{equation*}
		\item Para matrizes: fórmula de Dyson:
		      \begin{equation*}
		      	\exp(t(A+B)) = \exp(tA) + \int_0^t \exp((t-s)(A+B)) B \exp(sA)\, ds
		      \end{equation*}
	\end{itemize}
\end{frame}

%--------------------------------------------------------

\begin{frame}{Produto interno hermitiano}
	\begin{itemize}
		\item Produto interno unidimensional:
		      \begin{equation*}
		      	\langle u, v \rangle = \int_{-\infty}^{+\infty} \frac{e^{-x^2/2}}{\sqrt{2\pi}} u(x)v(x)\, dx
		      \end{equation*}
		\item Ortogonalidade: $\langle p_n, p_m \rangle = 0$ se $n \neq m$
		\item Normalização: $\langle p_n, p_n \rangle = 1$
	\end{itemize}
\end{frame}

\begin{frame}{Extensão $n$-dimensional}
	\begin{itemize}
		\item Produto interno generalizado:
		      \begin{equation*}
		      	\langle u, v \rangle = \int \cdots \int (2\pi)^{-n/2} e^{-\sum \frac{x_i^2}{2}} u(x)v(x)\, dx
		      \end{equation*}
		\item Possibilidade de definir polinômios ortonormais em $q$, $p$ com densidade $e^{-H/T}$
	\end{itemize}
\end{frame}

%--------------------------------------------------------

\begin{frame}{Projeção e Mori-Zwanzig}
	\begin{itemize}
		\item Espaço $\Gamma$ $n$-dimensional com densidade de probabilidade.
		\item Divide-se $x$ em $\hat{x}$ (resolvidas) e $\tilde{x}$ (não resolvidas).
		\item Projeção ortogonal: $\mathbb{P}g = \mathbb{E}[g \mid \hat{x}]$
		\item Subespaço gerado por polinômios hermitianos em $\hat{x}$.
	\end{itemize}
\end{frame}