\section{Motivação} % Seções são adicionadas para organizar sua apresentação em blocos discretos, todas as seções e subseções são automaticamente exibidas no índice como uma visão geral da apresentação, mas NÃO são exibidas como slides separados.

%----------------------------------------------------------------------------------------

\begin{frame}{Motivação: exemplo didático}
\begin{itemize}
    \item Exemplo de \cite{Chorin2013}.
    \item Sistema com duas partículas em 1D.
    \item Hamiltoniano:
    \begin{equation*}
        H = \frac{1}{2}(q_1^2 + q_2^2 + q_1^2 q_2^2 + p_1^2 + p_2^2)
    \end{equation*}
\end{itemize}
\end{frame}

\begin{frame}{Equações de movimento}
\begin{align*}
    \dot{q}_1 &= p_1, \quad &\dot{p}_1 &= -q_1(1 + q_2^2) \\
    \dot{q}_2 &= p_2, \quad &\dot{p}_2 &= -q_2(1 + q_1^2)
\end{align*}
\begin{itemize}
    \item Inicialmente, apenas $q_1(0)$ e $p_1(0)$ são conhecidos.
    \item $q_2(0)$ e $p_2(0)$ são amostrados de
    \begin{equation*}
        W = \frac{e^{-H(q,p)}}{Z}
    \end{equation*}
\end{itemize}
\end{frame}

\begin{frame}{Simulação e estimativa}
\begin{itemize}
    \item Para cada amostra de $q_2(0)$ e $p_2(0)$, obtemos uma nova trajetória de $q_1(t)$ e $p_1(t)$.
    \item Interesse em:
    \begin{equation*}
        \mathbb{E}[q_1(t)\mid q_1(0), p_1(0)], \quad \mathbb{E}[p_1(t)\mid q_1(0), p_1(0)]
    \end{equation*}
    \item Média feita sobre várias simulações.
\end{itemize}
\end{frame}

\begin{frame}{Limitação da abordagem}
\begin{itemize}
    \item A abordagem  é válida apenas para \textbf{tempos curtos}.
    \item Para $t$ grande: as médias se afastam dos valores reais.
\end{itemize}
\end{frame}