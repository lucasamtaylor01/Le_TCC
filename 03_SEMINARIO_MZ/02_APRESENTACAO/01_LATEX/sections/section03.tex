\section{Mori-Zwanzig}

\begin{frame}{Construção do formalismo MZ}
\begin{itemize}
    \item Separação das variáveis: $x = (\hat{x}, \tilde{x})$ e $\phi = (\hat{\phi}, \tilde{\phi})$ e $R = (\hat{R}, \tilde{R})$
    \item Foco nas $m$ primeiras componentes: variáveis resolvidas.
    \item Escrevemos: $\hat{\phi}_j(x,t) = e^{tL}x_j$, $1 \leq j \leq m$
    \item Derivando no tempo:
    \begin{equation*}
        \frac{\partial}{\partial t}e^{tL}x_j = e^{tL}Lx_j
    \end{equation*}
    \item Com projeções ortogonais: $\mathbb{Q} = I - \mathbb{P}$
\end{itemize}
\end{frame}

%--------------------------------------------------------

\begin{frame}{Equação de Mori-Zwanzig via Dyson}
\begin{itemize}
    \item Fórmula de Dyson aplicada:
    \begin{equation*}
        e^{tL} = e^{t\mathbb{Q}L} + \int_0^t e^{(t-s)L} \mathbb{P}L e^{s\mathbb{Q}L} \, ds
    \end{equation*}
    \item Substituindo na equação de evolução:
    \begin{equation*}
        \frac{\partial}{\partial t} e^{tL} x_j = \ e^{tL} \mathbb{P}L x_j + e^{t\mathbb{Q}L} \mathbb{Q}L x_j + \int_0^t e^{(t-s)L} \mathbb{P}L e^{s\mathbb{Q}L} \mathbb{Q}L x_j \, ds
    \end{equation*}
\end{itemize}
\end{frame}

%--------------------------------------------------------

\begin{frame}{Análise termo a termo}
    Vamos decompor o termo $e^{tL} x_j$ segundo a equação de Mori-Zwanzig, analisando separadamente:
    \begin{itemize}
        \item O termo markoviano (primeiro termo),
        \item O termo de ruído (segundo termo),
        \item O termo de memória (terceiro termo).
    \end{itemize}
\end{frame}

%--------------------------------------------------------

\begin{frame}{Primeiro termo}
O primeiro termo é dado por:
\begin{equation}
	e^{tL} \mathbb{P}L x_j
	\label{eq:primeiro-termo-mz}
\end{equation}

A partir dele, temos que:
\begin{align*}
	Lx_j &= \sum_i R_i\left(\frac{\partial}{\partial x_i}\right)x_j = R_j(x), \\
	\mathbb{P}Lx_j &= \mathbb{E}[R_j(x)\,|\,\hat{x}], \\
	e^{tL}\mathbb{P}Lx_j &= \bar{R}_j\left(\hat{\phi}(x,t)\right).
\end{align*}

\footnotesize{Note que é um termo markoviano, pois depende apenas do estado atual $\hat{\phi}(x,t)$.}
\end{frame}

%--------------------------------------------------------

\begin{frame}{Segundo termo}

O segundo termo é dado por:
\begin{equation*}
	w_j = e^{t\mathbb{Q}L} \mathbb{Q}L x_j
\end{equation*}

A partir dele, temos que:
\begin{align}
	\frac{\partial}{\partial t} w_j(x,t) &= \mathbb{Q}L w_j(x,t), \nonumber\\
	w_j(x,0) &= \mathbb{Q}L x_j = R_j(x) - \mathbb{E}[R_j \mid \hat{x}]. 
    \label{eq:mori-zwanzig-dinamica-ortogonal}
\end{align}

A função $w_j(x,0)$ representa a \textit{parte flutuante} de $R_j(x)$ e evolui segundo a dinâmica ortogonal. Temos $\mathbb{P} w_j(x,t) = 0$ para todo $t$.
\end{frame}

%--------------------------------------------------------

\begin{frame}{Subespaço de ruído}
O \textit{subespaço do ruído} é formado por funções ortogonais às funções de $\hat{x}$.

Isso corresponde, geralmente, a termos que dependem de $\tilde{x}$. Esses termos são imprevisíveis dado $\hat{x}$.
\end{frame}

\begin{frame}{Terceiro termo}

O terceiro termo é dado por:
\begin{equation*}
	\int_0^t e^{(t-s)L} \mathbb{P}L e^{s\mathbb{Q}L} \mathbb{Q}L x_j
\end{equation*}
Este termo depende do histórico do sistema: é o \textbf{termo de memória}.

Projeção usando polinômios hermitiano $H_1, H_2, \dots$:
\begin{align*}
	\mathbb{P}L e^{s\mathbb{Q}L} \mathbb{Q}L x_j &= \sum_k \langle L\mathbb{Q} e^{s\mathbb{Q}L} \mathbb{Q}Lx_j, H_k(\hat{x}) \rangle H_k(\hat{x}).
\end{align*}
\end{frame}

%--------------------------------------------------------

\begin{frame}{Produto interno e antissimetria de $L$}
Se $L$ é antissimétrico: $(u, Lv) = -(Lu, v)$, então:
\begin{align*}
	(L \mathbb{Q} e^{s \mathbb{Q} L} \mathbb{Q}L x_j, H_k) 
	  &= - (\mathbb{Q} e^{s \mathbb{Q} L} \mathbb{Q}L x_j, L H_k) \\
	  &= - (e^{s \mathbb{Q} L} \mathbb{Q}L x_j, \mathbb{Q}L H_k).
\end{align*}

O terceiro termo é uma soma de covariâncias temporais de ruído.
\end{frame}
