\documentclass[12pt]{article}
\usepackage[utf8]{inputenc}
\usepackage[a4paper, margin=2.5cm]{geometry}
\usepackage{lipsum}
\usepackage{parskip} % espaço entre parágrafos
\usepackage{setspace} % espaçamento
\usepackage{graphicx}
\usepackage{amsmath, amssymb}
\usepackage{enumitem}
\usepackage{biblatex}
\addbibresource{referencias.bib}

\begin{document}

\begin{titlepage}
    \centering
    {\Large\scshape MAP2419 -- Introdução ao Trabalho de Formatura \par}
    \vspace{0.3cm}
    {\large Instituto de Matemática e Estatística da Universidade de São Paulo (IME-USP)\par}
    
    \vspace{3cm}
    
    {\LARGE\bfseries Projeto\par}
    \vspace{1cm}
    {\LARGE\bfseries Uma abordagem estocástica do Modelo Lorenz 80\par}
    
    \vfill
    
    \begin{minipage}[t]{0.45\textwidth}
        \raggedright
        \textbf{Orientador:} \par
        Prof. Dr. Breno Raphaldini Ferreira da Silva \par
        \texttt{brenorfs@gmail.com} \par
        IME-USP
    \end{minipage}
    \hfill
    \begin{minipage}[t]{0.45\textwidth}
        \raggedright
        \textbf{Aluno:} \par
        Lucas Amaral Taylor \par
        NUSP: 13865062 \par
        \texttt{lucasamtaylor@usp.br} \par
        IME-USP
    \end{minipage}
    
    \vspace{2cm}
\end{titlepage}

\section*{Introdução}
\lipsum[1-2]

\section*{Objetivos}
\lipsum[1-2]

\section*{Metodologia}
\lipsum[1-2]

\section*{Cronograma}
\lipsum[1-2]

\nocite{*}
\printbibliography[heading=none]

\end{document}
