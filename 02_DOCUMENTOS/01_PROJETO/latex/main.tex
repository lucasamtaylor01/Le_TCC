\documentclass[12pt]{article}

% Codificação e idioma
\usepackage[utf8]{inputenc}
\usepackage[brazilian]{babel}

% Layout da página e espaçamento
\usepackage[a4paper, left=3cm, right=2cm, top=3cm, bottom=2cm]{geometry}
\usepackage{setspace}
\usepackage{parskip}

% Matemática
\usepackage{amsmath, amssymb}

% Gráficos e imagens
\usepackage{graphicx}

% Listas e tabelas
\usepackage{enumitem}
\usepackage{booktabs}

% Texto fictício
\usepackage{lipsum}

% Citações e bibliografia
\usepackage{csquotes}
\usepackage[
    style=authoryear-comp,
    backend=biber,
    maxcitenames=2,
    maxbibnames=99,
    uniquelist=false,
    dashed=false,
    doi=true,
    url=true,
    giveninits=true,
    uniquename=init,
    sorting=nyt,
    language=portuguese,
    natbib=true
]{biblatex}
\addbibresource{referencias.bib}



\begin{document}

\begin{titlepage}
    \centering
    {\Large\scshape MAP2419 -- Introdução ao Trabalho de Formatura \par}
    \vspace{0.3cm}
    {\large Instituto de Matemática e Estatística da Universidade de São Paulo (IME-USP)\par}
    
    \vspace{3cm}
    
    {\LARGE\bfseries Projeto\par}
    \vspace{1cm}
    {\LARGE\bfseries Um abordagem estocástica do\\ Modelo Lorenz 80\par}
    
    \vfill
    
    \begin{minipage}[t]{0.45\textwidth}
        \raggedright
        \textbf{Orientador:} \par
        Prof. Dr. Breno Raphaldini Ferreira da Silva  \par
        \texttt{brenorfs@gmail.com} \par
        \textit{IME-USP} \par\medskip
    \end{minipage}
    \hfill
    \begin{minipage}[t]{0.45\textwidth}
        \raggedright
        \textbf{Aluno:} \par
        Lucas Amaral Taylor \par
        NUSP: 13865062 \par
        \texttt{lucasamtaylor@usp.br} \par
        \textit{IME-USP}
    \end{minipage}
    \vspace{2cm}
\end{titlepage}

\section*{Introdução}
Em janeiro de 2025, o aluno Lucas Amaral Taylor cursou a disciplina \textbf{MAP5007 - Ondas em Fluidos Geofísicos}, oferecida no programa de verão do Instituto de Matemática e Estatística (IME-USP), ministrada pelo Prof. Dr. Breno Raphaldini Ferreira da Silva. A disciplina teve como objetivo apresentar conceitos básicos da dinâmica de fluidos geofísicos por meio de uma abordagem matemática \citep{uspJanus}.

Ao final da disciplina, o aluno realizou um seminário com o tema ``\textit{Um breve estudo do Modelo Lorenz 80}'', cujo objetivo foi apresentar os aspectos gerais do modelo geofísico desenvolvido por Edward Norton Lorenz no artigo ``\textit{Attractor Sets and Quasi-Geostrophic Equilibrium}'' \citep{Lorenz1980}. Os arquivos do seminário estão disponíveis publicamente no repositório do \textit{GitHub} \citep{TaylorL80}.

O presente trabalho de conclusão de curso é uma extensão desse estudo prévio. Em novembro de 2021, foi publicado o artigo ``\textit{Stochastic Rectification of Fast Oscillations on Slow Manifold Closures}'' \citep{Chekroun2021}, que propõe uma abordagem estocástica para sistemas lentos-rápidos, utilizando métodos da física estatística. Para tal, emprega-se o modelo de Lorenz 80 como caso de estudo e, particularmente, o método de Mori-Zwanzig, que, sucintamente, é um método da física estatística que separa a dinâmica de um sistema em partes relevantes e irrelevantes por meio de operadores de projeção.

Um dos aspectos fundamentais do método de Mori-Zwanzig é que, ao projetar a dinâmica de um sistema sobre um subespaço de variáveis relevantes, os efeitos das variáveis descartadas não desaparecem. Eles são incorporados na dinâmica efetiva na forma de dois termos adicionais: um termo markoviano, que representa a influência dos estados passados, e um termo de ruído, que traduz a variabilidade não resolvida. 

Neste trabalho, serão abordados o arcabouço teórico necessário para a compreensão do método de Mori-Zwanzig, sua construção e suas propriedades. Além disso, realizaremos simulações computacionais utilizando a linguagem Julia e, em especial, a biblioteca \textit{SciML: Open Source Software for Scientific Machine Learning}, que oferece ferramentas para o tratamento de equações diferenciais estocásticas. 

✨Como extensão do trabalho apresentado por \citet{Chekroun2021}, também exploraremos diferentes estruturas e características do ruído, investigando como sua modelagem impacta a dinâmica efetiva do modelo de Lorenz 80. A intenção é avaliar se abordagens alternativas ou complementares para a representação do ruído podem enriquecer ou aperfeiçoar os resultados obtidos na formulação estocástica do sistema.

Por fim, ressalta-se que este trabalho integra conhecimentos em análise, equações diferenciais ordinárias, equações diferenciais parciais, álgebra linear, estatística, física e computação. Esses tópicos estão diretamente alinhados com a formação do curso de Bacharelado em Matemática Computacional com habilitação em Métodos Matemáticos. Dessa forma, o trabalho de conclusão de curso configura-se como uma atividade de caráter transversal e interdisciplinar, abrangendo de maneira integrada os conteúdos desenvolvidos ao longo da graduação.


\section*{Objetivos}
\begin{enumerate}
    \item \lipsum[1][1-2]
    \begin{enumerate}
        \item \lipsum[2][1]
        \item \lipsum[2][2]
    \end{enumerate}
    \item \lipsum[1][1-2]
    \begin{enumerate}
        \item \lipsum[2][1]
        \item \lipsum[2][2]
    \end{enumerate}
    \item \lipsum[1][1-2]
    \begin{enumerate}
        \item \lipsum[2][1]
        \item \lipsum[2][2]
    \end{enumerate}
\end{enumerate}

\section*{Metodologia}
\lipsum[1-2]

\section*{Cronograma}
\begin{center}
\renewcommand{\arraystretch}{1.5}
\begin{tabular}{p{3cm}p{10cm}}
    \toprule
    \textbf{Mês} & \textbf{Atividade} \\
    \midrule
    Abril   & Definição de tema, orientador e seleção das principais referências.  \\
    Maio    & Introdução ao formalismo de Mori-Zwanzig; Introdução à linguagem Julia; \\
    Junho   & \lipsum[3][1] \\
    Julho   & \lipsum[4][1] \\
    Agosto  & \lipsum[5][1] \\
    Setembro& \lipsum[6][1] \\
    Outubro & Conclusão da monografia \\
    Novembro & Revisão, tradução e apresentação  \\
    \bottomrule
\end{tabular}
\end{center}
\newpage
\nocite{*}
\printbibliography[title={Referências}, label={sec:bib}]

\end{document}